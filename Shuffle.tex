\documentclass[11pt]{article}
\usepackage{newcent}
\usepackage{listings}
\usepackage{graphicx}
\usepackage{caption, float}
\usepackage{fdsymbol}


\oddsidemargin=0.0in
\evensidemargin=0.0in
\topmargin= -0.5in
\textheight=8.75in
\textwidth=6.5in
\parindent=25pt
\parskip=10pt
\baselineskip=16.5pt

%\nofiles this command will let you run latex without creating .aux,.bbl,etc
\begin{document}
\pagestyle{plain}



\begin{center}
  Shuffle Cards
  John Williams\\
\end{center}

A long time ago,\ myself and two friends were playing around with a deck of cards.  We asked
the question:  'If you shuffled a deck of cards perfectly,\ how many shuffles would it take to
get the cards back in the exact same order?'

We defined perfect shuffling as :

\begin{itemize}
\item[1.]  Split the deck exactly in half.
\item[2.]  Always start the interleaving with the first half of the split deck.
\item[3.] Interleave the cards from one half perfectly with the cards from the second half.
\end{itemize}

For example,\ suppose you have a deck of 4 cards,\ Cards [A$\!\clubsuit$, K$\!\clubsuit$,
Q$\!\clubsuit$, J$\!\clubsuit$].  Splitting gives two decks of [A$\!\clubsuit$, K$\!\clubsuit$]
and [Q$\!\clubsuit$, J$\!\clubsuit$].  Shuffling(interleaving) gives a new deck
[A$\!\clubsuit$, Q$\!\clubsuit$, K$\!\clubsuit$, J$\!\clubsuit$].  Splitting again gives two
new decks [A$\!\clubsuit$, Q$\!\clubsuit$] and [K$\!\clubsuit$, J$\!\clubsuit$].  Interleaving
once more gives [A$\!\clubsuit$, K$\!\clubsuit$, Q$\!\clubsuit$, J$\!\clubsuit$] and we are
back to the original sequence.  Therefore it took 2 shuffles to get back to the original
sequence.

This example made us realize that we could do the same thing for a 'deck' of any size and ask
the question : 'How many shuffles does it take to get back the original sequence?'  As an
aside,\ it was found that decks with odd numbers of cards took the exact same number of
shuffles as the even length deck just prior to it.  Therefore,\ odd length card decks were not
evaluated. 

The following selection of plots shows the results for sequences of various lengths.  There are
some interesting features.  Clearly,\ no deck needs more shuffles then number of values in the
original deck.  For some decks,\ though,\ the number of shuffles seems remarkably small.  Since
the splitting and shuffling is in a sense a binary operation(split exactly in half, shuffle
exactly) it make sense that the decks with powers of 2 length would have shuffle values that
are those powers of two.  For instance,\ a deck of 256($2^{8}$) cards takes 8 shuffles to get
back to the original sequence.  However,\ it is surprising that decks of length $2^{n} + 1$
take exactly $2^{n} + 1$ shuffles,\ but decks of length $2^{n} + 2$ do not take $2^{n} + 2$
shuffles.

Some other features are indicated on the plots.  Clearly there is a set of shuffles that when
plotted have shuffles that lie on a line of slope $1/2$ and also a line of slope $1/3$.  



\end{document}
